% LMU Exam Package
% Note: xcolor, amsmath, amssymb are already loaded by Quarto
\usepackage{fancyhdr}
\usepackage{lastpage}
\usepackage{tikz}

%% Solution mode toggle - will be set by Lua filter
\newif\ifsolution
\solutionfalse

%% Metadata commands - defaults, will be redefined by Lua filter
\newcommand{\examsemester}{}
\newcommand{\examveranstaltung}{}
\newcommand{\examveranstaltungkurz}{}  % Short name for header
\newcommand{\examdozent}{}
\newcommand{\examdatum}{}
\newcommand{\examdauer}{}
\newcommand{\anzahlaufgaben}{0}  % Total number of exercises
\newcommand{\gesamtpunkte}{0}    % Total points across all exercises

%% Points command - displays points in parentheses
\newcommand{\punkte}[1]{\hfill\textbf{(#1 Punkte)}}

%% Answer field with 5mm grid (only shown when solution=false)
%% Usage: \antwortfeld{height in cm}, e.g., \antwortfeld{4}
\newcommand{\antwortfeld}[1]{%
  \ifsolution
    \vspace{0.5\baselineskip}% minimal space in solution mode
  \else
    \par\vspace{0.3\baselineskip}%
    \begin{tikzpicture}
      \draw[step=5mm, gray!40, thin] (0,0) grid (\textwidth, #1);
      \draw[gray!60] (0,0) rectangle (\textwidth, #1);
    \end{tikzpicture}%
    \par\vspace{0.3\baselineskip}%
  \fi
}

%% Legacy command - kept for compatibility
\newcommand{\leerzeile}[1][1]{%
  \par\vspace{#1\baselineskip}%
}

%% Points table - will be redefined by Lua filter
\newcommand{\punktetabelle}{%
\begin{tabular}{|l|c|c|}
  \hline
  & mögliche Punkte & erreichte Punkte \\
  \hline
  \hline
  \textit{(Keine Punkte definiert)} & & \\
  \hline
  \textbf{Summe} & & \\
  \hline
\end{tabular}%
}

%% Header and footer
\pagestyle{fancy}
\fancyhf{}
\renewcommand{\headrulewidth}{0.4pt}
\renewcommand{\footrulewidth}{0.4pt}
\fancyhead[L]{\small Klausur \examsemester{} --- \examveranstaltungkurz}
\fancyhead[R]{\small Name: \rule{4cm}{0.4pt}}
\fancyfoot[R]{Seite \thepage{} von \pageref{LastPage}}
