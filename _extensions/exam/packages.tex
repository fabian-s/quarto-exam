% LMU Exam Package
% Note: xcolor, amsmath, amssymb are already loaded by Quarto
\usepackage{fancyhdr}
\usepackage{lastpage}
\usepackage{tikz}
\usepackage[most]{tcolorbox}

%% Solution box style - dark gray left border
\newtcolorbox{solutionbox}{
  blanker,
  left=4mm,
  borderline west={3pt}{0pt}{gray!50},
  before skip=0.5\baselineskip,
  after skip=0.5\baselineskip,
  boxsep=2mm,
  arc=0pt,
  outer arc=0pt,
  top=2mm,
  bottom=2mm,
  right=2mm,
}

%% Solution mode toggle - defined and set by Lua filter in header-includes
%% \ifsolution is defined by the filter before this file is included

%% Grid mode toggle - defined by Lua filter
%% \ifexamgrid controls whether answer fields have grid lines or are blank
\providecommand{\examgridtrue}{\global\examgridtrue}
\providecommand{\examgridfalse}{\global\examgridfalse}

%% Metadata commands - defined by Lua filter (providecommand as fallback)
\providecommand{\examsemester}{}
\providecommand{\examcourse}{}
\providecommand{\examcourseshort}{}
\providecommand{\examinstructor}{}
\providecommand{\examdate}{}
\providecommand{\examduration}{}
\providecommand{\examexercisecount}{0}
\providecommand{\examtotalpoints}{0}

%% Points command - displays points in parentheses
\newcommand{\exampoints}[1]{\hfill\textbf{(#1 Punkte)}}

%% Point markers for grading - only visible in solution mode
%% \p = 1 point, \hp = 0.5 points (half), \pp = 2 points (double)
%% These work both in text mode and math mode
\newcommand{\pmark}[1]{\ifsolution\ifmmode\text{\textsuperscript{\textcolor{red}{\tiny[#1]}}}\else\textsuperscript{\textcolor{red}{\tiny[#1]}}\fi\fi}
\newcommand{\p}{\pmark{1P}}
\newcommand{\hp}{\pmark{½P}}
\newcommand{\pp}{\pmark{2P}}

%% Answer field (only shown when solution=false)
%% Uses breakable tcolorbox so it can continue onto next page
%% Grid lines shown only when \ifexamgrid is true
%% Usage: \examanswerfield{height in cm}, e.g., \examanswerfield{4}
\newlength{\examanswerfieldtarget}
\newlength{\examanswerfieldcurrent}
\newcommand{\examanswerfield}[1]{%
  \ifsolution
    \vspace{0.5\baselineskip}% minimal space in solution mode
  \else
    \par\vspace{0.3\baselineskip}%
    \ifexamgrid
      % Grid version
      \begin{tcolorbox}[
        breakable,
        enhanced jigsaw,
        colback=white,
        colframe=gray!40,
        boxrule=0.4pt,
        arc=0pt,
        outer arc=0pt,
        left=0pt, right=0pt, top=0pt, bottom=0pt,
        boxsep=0pt,
        pad at break=0pt,
        underlay unbroken and first={
          \begin{tcbclipinterior}
            \draw[step=5mm, gray!40, thin] (interior.south west) grid (interior.north east);
          \end{tcbclipinterior}
        },
        underlay middle={
          \begin{tcbclipinterior}
            \draw[step=5mm, gray!40, thin] (interior.south west) grid (interior.north east);
          \end{tcbclipinterior}
        },
        underlay last={
          \begin{tcbclipinterior}
            \draw[step=5mm, gray!40, thin] (interior.south west) grid (interior.north east);
          \end{tcbclipinterior}
        },
      ]
    \else
      % Blank version (just grey border)
      \begin{tcolorbox}[
        breakable,
        enhanced jigsaw,
        colback=white,
        colframe=gray!40,
        boxrule=0.4pt,
        arc=0pt,
        outer arc=0pt,
        left=0pt, right=0pt, top=0pt, bottom=0pt,
        boxsep=0pt,
        pad at break=0pt,
      ]
    \fi
    % Insert 1cm tall paragraphs to allow page breaks
    % Use dimension comparison to handle decimal heights (e.g., 7.5)
    \setlength{\examanswerfieldtarget}{#1 cm}%
    \setlength{\examanswerfieldcurrent}{0pt}%
    \loop\ifdim\examanswerfieldcurrent<\examanswerfieldtarget
      \strut\par\vspace{0.65cm}%
      \addtolength{\examanswerfieldcurrent}{1cm}%
    \repeat
    \end{tcolorbox}%
    \par\vspace{0.3\baselineskip}%
  \fi
}

%% Points table - defined by Lua filter (providecommand as fallback)
\providecommand{\exampointstable}{%
\begin{tabular}{|l|c|c|}
  \hline
  & mögliche Punkte & erreichte Punkte \\
  \hline
  \hline
  \textit{(Keine Punkte definiert)} & & \\
  \hline
  \textbf{Summe} & & \\
  \hline
\end{tabular}%
}

%% Header and footer
\pagestyle{fancy}
\fancyhf{}
\renewcommand{\headrulewidth}{0.4pt}
\renewcommand{\footrulewidth}{0.4pt}
\fancyhead[L]{\small Klausur \examsemester{} --- \examcourseshort}
\fancyhead[R]{}
\fancyfoot[R]{Seite \thepage{} von \pageref{LastPage}}
